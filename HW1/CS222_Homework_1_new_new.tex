%!TEX program = Xelatex
\documentclass{article}
%\usepackage{ctex}
\usepackage{amsmath,amscd,amsbsy,amssymb,latexsym,url,bm,amsthm}
\usepackage{epsfig,graphicx,subfigure}
\usepackage{enumitem,balance,mathtools}
\usepackage{wrapfig}
\usepackage{mathrsfs, euscript}
\usepackage[usenames]{xcolor}
\usepackage{hyperref}
\usepackage{caption}
%\usepackage{subcaption}
\usepackage{float}
\usepackage{listings}
%\usepackage{enumerate}
%\usepackage{algorithm}
%\usepackage{algorithmic}
%\usepackage[vlined,ruled,commentsnumbered,linesnumbered]{algorithm2e}
\usepackage[ruled,lined,boxed,linesnumbered]{algorithm2e}

\newtheorem{theorem}{Theorem}[section]
\newtheorem{lemma}[theorem]{Lemma}
\newtheorem{proposition}[theorem]{Proposition}
\newtheorem{corollary}[theorem]{Corollary}
\newtheorem{exercise}{Exercise}[section]
\newtheorem*{solution}{Solution}

\renewcommand{\thefootnote}{\fnsymbol{footnote}}

\newcommand{\postscript}[2]
    {\setlength{\epsfxsize}{#2\hsize}
    \centerline{\epsfbox{#1}}}

\renewcommand{\baselinestretch}{1.0}

\setlength{\oddsidemargin}{-0.365in}
\setlength{\evensidemargin}{-0.365in}
\setlength{\topmargin}{-0.3in}
\setlength{\headheight}{0in}
\setlength{\headsep}{0in}
\setlength{\textheight}{10.1in}
\setlength{\textwidth}{7in}

\title{CS222 Homework 1}
\author{Algorithm Analysis \& Deadline: 2020-09-21 Monday 24:00}
\date{Exercises for Algorithm Design and Analysis by Li Jiang, 2020 Autumn Semester}

\begin{document}

\maketitle

\begin{enumerate}

\item Prove that $\log (\log n) = o(n^k)$, where k is a positive constant. (ps: $\log n$ refers to $\log_2 n$.)

~\\
~\\

\item Prove that for any integer $n^2 -1\;\textgreater\;3$, there is a prime $p$ satisfying $n!\;\textgreater\;p\;\textgreater n$.

~\\
~\\
\item Assume that there is a recurrence formula as follows: 
\begin{equation*}
	D(x) = \begin{cases}
	1, &if\;\lfloor x \rfloor \leq 1\\
	3D(x/4) + x - 2, &if\;\lfloor x \rfloor  > 1
	\end{cases}
\end{equation*}
Please deduce the non-recursive expression of $D(x)$ and point out its asymptotic complexity. 

~\\
~\\


\item Use the minimal counterexample principle to prove that for any integer $n \textgreater 10$, there exist integers $i_n\geq0$ and $j_n\geq 0$, such that $n = i_n \times 3 + j_n \times 4$.
~\\
~\\

\item  Analyze the \textbf{average} time complexity of QuickSort in Alg.~\ref{Alg_Quick}.

    \begin{minipage}[t]{0.8\textwidth}
    \begin{algorithm}[H]
      \KwIn{An array $A[1,\cdots,n]$}
      \KwOut{$A[1,\cdots,n]$ sorted nondecreasingly}

      \BlankLine
      \caption{QuickSort}\label{Alg_Quick}

      %\If{$n \le 1$}{
      %  \Return\;
      %}

      $pivot \leftarrow A[n]$; $i \leftarrow 1$\;
      \For{$j \leftarrow 1$ \KwTo $n-1$}{
        \If{$A[j] < pivot$}{
          swap $A[i]$ and $A[j]$\;
          $i \leftarrow i+1$\;
        }
      }

      swap $A[i]$ and $A[n]$\;
      \lIf{$i>1$}{$\operatorname{QuickSort}(A[1,\cdots,i-1])$}
      \lIf{$i<n$}{$\operatorname{QuickSort}(A[i+1,\cdots,n])$}
    \end{algorithm}
    \end{minipage}
    
~\\
~\\
\newpage
\item Rank the following functions by order of growth with explanations: that is, find an arrangement $g_1, g_2, \ldots , g_{k}$ of the functions $g_1 = \Omega(g_2), g_2 = \Omega(g_3), \ldots, g_{k-1} = \Omega(g_{k})$.  Partition your list into equivalence classes such that functions $f(n)$ and $g(n)$ are in the same class if and only if $f(n) = \Theta(g(n))$. Use symbols ``$=$'' and ``$\prec$'' to order these functions appropriately. (ps: $\log n$ refers to $\log_2 n$.)
    $$
    \begin{array}{ccccc}
        2^{\log n} \quad & \quad (\log n)^{\ln n} \quad & \quad n^2 \quad & \quad n! \quad & \quad (n - 1)! \\
        2^n \quad & \quad n^3 \quad & \quad \log^2 n \quad & \quad e^n \quad & \quad 2^{2^n} \\
        \log\log n \quad & \quad (n+1)\cdot 2^n \quad & \quad n \quad & \quad \log {(n^2 - n)} \quad & \quad 2^{\ln n} \\
    \end{array}
    $$
    
~\\
~\\


\textbf{Remark}: You need to upload your .pdf file.
\end{enumerate}

\end{document}
